









% Beni Madhav Agrawal, Second Year 2 page Tech Resume





\documentclass[a4paper,10pt]{article}
%-----------------------------------------------------------
\usepackage[top=0.75in, bottom=0.75in, left=0.55in, right=0.85in]{geometry}
\usepackage{graphicx}
\usepackage{url}
\usepackage{palatino}
\usepackage{tabularx}
\usepackage{hyperref}
\fontfamily{SansSerif}
\selectfont

\usepackage[T1]{fontenc}
\usepackage
%[ansinew]
[utf8]
{inputenc}

\usepackage{color}
\definecolor{mygrey}{RGB}{188,170,164}
\definecolor{mywhite}{RGB}{255,255,255}
\textheight=9.75in
\raggedbottom

\setlength{\tabcolsep}{0in}
\newcommand{\isep}{-2 pt}
\newcommand{\lsep}{-0.5cm}
\newcommand{\psep}{-0.6cm}
\newcommand{\blank}[1]{\hspace*{#1}}
\renewcommand{\labelitemii}{$\circ$}

\pagestyle{empty}
%-----------------------------------------------------------
%Custom commands
\newcommand{\resitem}[1]{\item #1 \vspace{-2pt}}
\newcommand{\resheading}[1]{{\small \colorbox{mygrey}{\begin{minipage}{0.975\textwidth}{\textbf{#1 \vphantom{p\^{E}}}}\end{minipage}}}}
\newcommand{\resheadings}[1]{{\small \colorbox{mywhite}{\begin{minipage}{0.975\textwidth}{\textbf{#1 \vphantom{p\^{E}}}}\end{minipage}}}}
\newcommand{\ressubheading}[3]{
\begin{tabular*}{6.62in}{l @{\extracolsep{\fill}} r}
	\textsc{{\textbf{#1}}} & \textsc{\textit{[#2]}} \\
\end{tabular*}\vspace{-8pt}}
%-----------------------------------------------------------

\begin{document}
\hspace{0.5cm}\\[-0.2cm]

\indent  \\

\resheadings{}\\[\lsep]
\\ \\
%\begin{table}[ht!]
%\begin{center}
\\ \\ \\ 
\\ \\
\indent \indent Pursuing Minor in \textbf{Computer Sciences and Engineering} \\
\indent \indent GitHub - \href{https://github.com/gimmepizza}{https://github.com/gimmepizza}
%\end{center}
%\end{table}
\\

\resheading {\textbf{Skills} }\\[\lsep]
\begin{itemize}
\item \textbf{Languages:} C/C++, Python (with OpenCV), Basic Java (with Android), Arduino, \LaTeX
\item \textbf{Tools:} NgSpice, AutoCAD, Git, MATLAB
\item \textbf{Hardware:} Arduino, HC-05, Raspberry pi 3, TTL IC's, ATTINY2313 microprocessor.

\end{itemize}

\resheading{\textbf{ACADEMIC ACHIEVEMENTS} }\\[\lsep]
\begin{itemize}
\setlength\itemsep{0.05em}
\item Won the \textbf {Kishore Vaigyanik Protsahan Yojana '14 (KYPY)} Fellowship Award.
\item \noindent Recipient of National Talent Search Scholarship \textbf{2012} (\textbf{NTSE}) meant to identify and nurture talent in High School.

\item \noindent Achieved \textbf{AIR 578} (out of 0.2 million candidates) in \textbf{JEE Advance 2016}.
\item \noindent Acquired \textbf{99.71 percentile} (out of 1.2 million candidates) in \textbf{JEE Mains 2016}.

\item \noindent Awarded \textbf{Certificate of Merit} for getting into \textbf{Nation wise Top 1\%} in the preliminaries for the \textbf{Indian National Physics Olympiad (2015-16)}.

\item \noindent Awarded \textbf{Certificate of Merit} for getting into \textbf{State wise Top 1\%} in the preliminaries for the \textbf{Indian National Chemistry Olympiad (2015-16)}.


\end{itemize}

\resheading{\textbf{AWARDS AND SCHOLARSHIPS} }\\[\lsep]
\begin{itemize}
\setlength\itemsep{0.05em}

\item \noindent Finished as the \textbf{1\textsuperscript{st}Runner-Up} in \textbf{Discover India with Anupam Kher(2012)} an on air \textbf{television quiz show} and was awarded a Cash Prize of \textbf{INR 2,00,000}.

\item \noindent Awarded \textbf{Certificate of Recognition} from \textbf{Duke University} for successfully completing Talent Identification Program in India program for academically gifted students (\textbf{1 among 60 students all over India}) held at \textbf{Infosys Global Education Center ,Mysore}. \textbf{(2011)}

\item \noindent Stood \textbf {3\textsuperscript{rd}} in the country in \textbf {Technothlon, Techniche '12, IIT Guwahati}.

\item Recieved certificate of \textbf {High Distinction} for perfoming exceptionally in \textbf{ANCQ (Australian National Chemistry Quiz)} for the years \textbf{2010, 2011, 2012, 2013 and 2014}.

\end{itemize}

\resheading{\textbf{WORK EXPERIENCE} }\\[\lsep]
\begin{itemize}
\item \textbf{XLR8 Mentor}  \\
 \emph{(\textbf{Electronics Club-},  IIT Bombay)} \hfill August '17\\[-0.6cm]
	\begin{itemize}\itemsep \isep
	\item Managing the participants and supervising the smooth flow of the technical competition, \textbf{XLR8} ,it is a technical project held for freshers,which includes making a phone controlled car.
	\item Mentoring requires deep understanding of the circuit and good debugging skill.
	\end{itemize}


\end{itemize}


\resheading{\textbf{PROJECTS} }\\[\lsep]
\begin{itemize}
\item  \textbf{RoLED Display}  \hfill \hfill  \textsl{Institute Technical Summer Project}\\
 \emph{Electronics Club} \hfill \hfill Summer '17\\
 \href{https://drive.google.com/file/d/0B_04txdyD_zEdDMxbTU5YWpZSnM/view?usp=sharing}{Documentation :https://drive.google.com/file/d/0B_04txdyD_zEdDMxbTU5YWpZSnM/view?usp=sharing}
 \begin{itemize}\itemsep \isep
 \item Designed a \textbf{700 LED display using 16 LEDs} by synchronously rotating an array of LED which was interfaced with \textbf{Arduino}.
 \item Used this \textbf{Rotating LED display} along with \textbf{Bluetooth transmission} of images from the computer terminal to dynamically display images and small animations.
\item To display images , the image data was parsed to fit a circular array.
 \end{itemize}
\item \textbf{Shift and Add Multiplier} \\
 \emph{ Prof. M.B.Patil (Course Project)} \hfill {\em \emph{Feb'17 to Mar'17}}\\
 \href{https://youtu.be/aQzDZA82Cog}{Youtube Video: https://youtu.be/aQzDZA82Cog}
 \begin{itemize}\itemsep \isep
 \item Developed a Shift and Add Multiplier using TTL IC's that is without using any  Micro controller.
 \item Used \textbf{Logic Gates(7408,7432)}, \textbf{Adder(7483)} and \textbf{Shift Registers(7495)} for its working.
 \item Displayed the final result using LED's as a user interface.
 \end{itemize} 
\end{itemize}

\resheading{\textbf{LEADERSHIP AND RESPONSIBILITY} }\\[\lsep]
\begin{itemize}

\item \textbf{Electrical Engineering Students' Association, IIT Bombay - \emph{Joint Secretary}
}  \hfill { Mar'17 - Present} \\[-0.6cm]
	\begin{itemize}\itemsep \isep
	\item Coordinating with Faculty members and all Electrical Engineering students to conduct department events including excursion.

	\end{itemize}
\end{itemize}

\resheading{\textbf{KEY COURSES} }\\[\lsep]
\begin{itemize}
\setlength\itemsep{0.1em}
\item \noindent \textbf{Electrical Engineering} - Electronic Devices, Network Theory, Data Analysis and Interpretation, Introduction to Electrical Systems, Introduction to Electronics
\item \noindent \textbf{Computer Sciences} - Computer Networks, Basic Programming and Utilization
\item \noindent \textbf{Mathematics} - Complex Analysis, Calculus, Linear Algebra, Differential Equations
\item \noindent \textbf{Miscellaneous} - Quantum Physics and Application, Electricity and Magnetism

\end{itemize}

\resheading{\textbf{EXTRA CURRICULAR} }\\[\lsep]
\begin{itemize}
\setlength\itemsep{0.1em}
\item \textbf {Violin} :- Playing \textbf{since 2014} trained in school for 3 years, also completed NSO violin .
\item \textbf {Interests} :- Chess, Reading novels, Music and Solving Puzzles, Riddles.

\end{itemize}

\end{document}
Chat conversation end
Type a message...

Choose Files
Choose Files
